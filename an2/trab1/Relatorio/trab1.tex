% ------------------------------------------------------------------------
% ------------------------------------------------------------------------
% Relatório de Trabalho 1 de Algoritmos Numéricos 2
% Autor: André Barreto
% ------------------------------------------------------------------------
% ------------------------------------------------------------------------

\documentclass[
	11pt,				% tamanho da fonte
	oneside,			% para impressão apenas no verso. Oposto a twoside
	a4paper,			% tamanho do papel. 
	english,			% idioma adicional para hifenização
	brazil,				% o último idioma é o principal do documento
	]{article}


% ---
% PACOTES
% ---
\usepackage{lmodern}			% Usa a fonte Latin Modern
\usepackage[T1]{fontenc}		% Selecao de codigos de fonte.
\usepackage[utf8]{inputenc}		% Codificacao do documento (conversão automática dos acentos)
\usepackage{indentfirst}		% Indenta o primeiro parágrafo de cada seção.
\usepackage{nomencl} 			% Lista de simbolos
\usepackage{color}				% Controle das cores
\usepackage{graphicx}			% Inclusão de gráficos
\usepackage{microtype} 			% para melhorias de justificação
\usepackage{makeidx}			% Gerar índice
\usepackage{multirow,tabularx}
\usepackage{multicol}
\usepackage{listings}			% Include the listings-package
\usepackage{color}
\usepackage{hyperref}
\usepackage{cite}
\usepackage{url}
\usepackage[brazilian]{babel}
\usepackage[brazilian,hyperpageref]{backref}	 % Paginas com as citações na bibl
\usepackage{amsmath}
\usepackage{mathtools,amssymb}
\usepackage{pgfplots}

\usepackage{lipsum}
% ---

% ---
% Configurações do pacote listings
% ---
\definecolor{mygreen}{rgb}{0,0.6,0}
\definecolor{mygray}{rgb}{0.571428571,0.571428571,0.571428571}
\definecolor{mymauve}{rgb}{0.5714285718,0,0.82}
\definecolor{codegreen}{rgb}{0,0.6,0}
\definecolor{codegray}{rgb}{0.571428571,0.571428571,0.571428571}
\definecolor{codepurple}{rgb}{0.5714285718,0,0.82}
\definecolor{backcolour}{rgb}{0.95,0.95,0.92}
\lstset{
	backgroundcolor=\color{white},   % choose the background color; you must add \usepackage{color} or \usepackage{xcolor}
	basicstyle=\footnotesize,        % the size of the fonts that are used for the code
	breakatwhitespace=false,         % sets if automatic breaks should only happen at whitespace
	breaklines=true,                 % sets automatic line breaking
	captionpos=b,                    % sets the caption-position to bottom
	commentstyle=\color{mygreen},    % comment style
	deletekeywords={...},            % if you want to delete keywords from the given language
	escapeinside={\%*}{*)},          % if you want to add LaTeX within your code
	extendedchars=true,              % lets you use non-ASCII characters; for 8-bits encodings only, does not work with UTF-8
	keepspaces=true,                 % keeps spaces in text, useful for keeping indentation of code (possibly needs columns=flexible)
	keywordstyle=\color{blue},       % keyword style
	language=C,                      % the language of the code
	otherkeywords={*,...},           % if you want to add more keywords to the set
	numbers=left,                    % where to put the line-numbers; possible values are (none, left, right)
	numbersep=5pt,                   % how far the line-numbers are from the code
	numberstyle=\tiny\color{mygray}, % the style that is used for the line-numbers
	rulecolor=\color{black},         % if not set, the frame-color may be changed on line-breaks within not-black text (e.g. comments (green here))
	showspaces=false,                % show spaces everywhere adding particular underscores; it overrides 'showstringspaces'
	showstringspaces=false,          % underline spaces within strings only
	showtabs=false,                  % show tabs within strings adding particular underscores
	stepnumber=1,                    % the step between two line-numbers. If it's 1, each line will be numbered
	stringstyle=\color{mymauve},     % string literal style
	tabsize=2,	                   % sets default tabsize to 2 spaces
	title=\lstname                   % show the filename of files included with \lstinputlisting; also try caption instead of title
}
% ---
	
% ---
% Informações de dados para CAPA
% ---
\title{\textbf{Solução de Problemas de Valor no Contorno Bidimensionais}}
\author{
André Barreto e Igor Ventorim\\\\
\normalsize Universidade Federal do Espírito Santo\\
}
\date{2015}
% ---

% ---
% Configurações de aparência do PDF final
% ---
\definecolor{blue}{RGB}{41,5,195}

% informações do PDF
\makeatletter
\hypersetup{
	pdftitle={\@title}, 
	pdfauthor={\@author},
	pdfsubject={Escalonamento de Jobs},
	pdfcreator={LaTeX with abnTeX2},
	pdfkeywords={abnt}{latex}{abntex}{abntex2}{atigo científico}, 
	colorlinks=true,
	linkcolor=blue,
	citecolor=blue,
	filecolor=magenta,
	urlcolor=blue,
	bookmarksdepth=4
}
\makeatother
% --- 

% ---
% Compila o indice
% ---
\makeindex
% ---

% --- 
% Espaçamentos entre linhas e parágrafos 
% --- 
\setlength{\parindent}{1cm}
\setlength{\parskip}{0.2cm}
% ---

% ----
% Início do documento
% ----
\begin{document}
% ---

% Seleciona o idioma do documento
\selectlanguage{brazil}

% Retira espaço extra obsoleto entre as frases.
\frenchspacing

\graphicspath{ {Imagens/} }

% ----------------------------------------------------------
% Página de Título
% ----------------------------------------------------------
\begin{titlepage}
	\centering
	{\scshape \large Universidade Federal do Espírito Santo\par}
	{\large Departamento de Informática\par}
	\vspace{1cm}
	{\large André Barreto\par}
	
	\vfill
	
	{\LARGE \bfseries Método das Diferenças Finitas Aplicado a
Problemas Bidimensionais\par}
	\vspace{1cm}
	{\large Trabalho 1 de Algoritmos Numéricos 2\par}

	\vfill

	{\large Vitória\par}
	{\large 2016\par}
\end{titlepage}
\addtocounter{page}{1}

% ----------------------------------------------------------
% Introdução
% ----------------------------------------------------------
\section{Introdução}
O estudo da equação de transporte, também denominada equação
da advecção-difusão-reação, continua sendo um ativo campo de
pesquisa, uma vez que essa equação é de fundamental
importância nos problemas relacionados a aerodinâmica, 
meteorologia, oceanografia, hidrologia, engenharia química e
de reservatórios. A equação de transporte tem
características bastante peculiares que fazem com que sua 
resolução por meios numéricos seja dificultada em situações
onde o problema é fortemente convectivo. Por isso, diversos
métodos têm sido desenvolvidos e aplicados, com a intenção
de superar as dificuldades numéricas impostas por esta
equação.

A equação de transporte bidimensional pode ser definida por:

\begin{equation} \label{eq:transporte}
- k \left(\frac{\partial^2 u}{\partial x^2} + \frac{\partial^2 u}{\partial 
y^2}\right) +
\beta_x(x,y)\frac{\partial u}{\partial x} +
\beta_y(x,y)\frac{\partial u}{\partial y} +
\gamma(x,y)u = f(x,y) \text{ em } \Omega
\end{equation}

Este trabalho tem como objetivo verificar como as formas de
armazenamento das estruturas resultantes pela discretização
da equação \eqref{eq:transporte} por diferenças finitas pode
impactar no tempo de processamento.

Para isto, será implementado em linguagem C um programa capaz
de aplicar a discretização da equação de transporte e resolver
o sistema resultante discreto. Quanto ao método de resolução,
será aplicado o algoritmo SOR
(\textit{Sucessive Over Relaxation}) de duas formas:
resolvendo o sistema penta-diagonal armazenado de forma
esparsa, utilizando  cinto vetores; e completamente livre
de matriz.

Nas próximas seções estão descritos em detalhes o
desenvolvimento, testes e conceitos do trabalho, sendo que
o próximo tópico faz um resumo do método das Diferenças
Finitas, descrevendo as técnicas e ordem de aproximação
usadas.

% ----------------------------------------------------------
% Método das Diferenças Finitas
% ----------------------------------------------------------
\section{Método das Diferenças Finitas}
\lipsum[2]

% ----------------------------------------------------------
% Implementação
% ----------------------------------------------------------
\section{Implementação}
Nesta seção estão apresentados os códigos em C relevantes
do sistema que mostra suas principais funcionalidades
necessárias para a resolução do Método das Diferenças
Finitas em 2D.

% ----------------------------------------------------------
% Experimentos Numéricos
% ----------------------------------------------------------
\section{Experimentos Numéricos}
Para verificar a qualidade dos métodos implementados, foram
realizados uma séries de testes a um conjunto de
experimentos propostos. Em cada experimento, o programa em
questão é aplicado em diferentes ordens de sistema e em ambas
as versões do SOR apresentadas, onde o foco é a comparação dos
tempos de execução em cada caso.

Para isto, em cada experimento a seguir será ilustrado uma
tabela com os casos relevantes de teste e o tempo de
execução em segundos que cada computação levou. Os tempos
de execução apresentados foram recolhidos utilizando o
utilitário \textit{time} do Linux que mede o tempo que o
processo ficou rodando.

Para simplificar, algumas abreviações foram utilizadas nestas
tabelas, a saber:
\begin{itemize}
 \item $n$: partições no eixo X;
 \item $m$: partições no eixo Y;
 \item SOR ``normal'': algoritmo SOR utilizando a estrutura de
 armazenado da matriz pentadiagonal;
 \item SOR ``livre'': algoritmo SOR livre de matriz.
\end{itemize}

Os testes nesta seção foram executados em uma máquina com as seguintes configurações:

Ubuntu 14.04 64-bit \\
\indent Intel Core i7-3770 CPU @ 3.40GHz x 4 \\
\indent 8GB de memória


\subsection{Validação 1 - Problema simples com solução trivial}
Este é um experimento simples para testes do sistema onde
deve-se determinar a distribuição de calor em uma chapa de
metal, com faces termicamente isoladas e com espessura
desprezível, sendo que a temperatura é conhecida em todas
as faces da chapa. Neste caso, a equação \eqref{eq:transporte}
é dada por:

\begin{equation} \label{eq:v1}
- \left(\frac{\partial^2 u}{\partial x^2} + \frac{\partial^2 u}{\partial 
y^2}\right) = 0 \text{ em } \Omega
\end{equation}

Sendo $T_0$ a temperatura nas faces, espera-se que os valores
no interior da placa sejam igualmente $T_0$ em todos os pontos
da discretização.

Veja na tabela \ref{tab:tv1} os tempos de execução deste experimento
por valores $n$ e $m$ e versão do algoritmo SOR.

\begin{table}[ht]
\centering
\begin{tabular}{|c|c|l|c|}
\hline 
\textbf{n} & \textbf{m} & \textbf{SOR} & \textbf{Tempo} \\
\hline
\multirow{2}{*}{25}    & \multirow{2}{*}{100}  & normal & OOOOOOO \\
                       &                       & livre  & OOOOOOO \\
\hline
\multirow{2}{*}{100}   & \multirow{2}{*}{100}  & normal & OOOOOOO \\
                       &                       & livre  & OOOOOOO \\
\hline
\multirow{2}{*}{500}   & \multirow{2}{*}{1000} & normal & OOOOOOO \\
                       &                       & livre  & OOOOOOO \\
\hline
\multirow{2}{*}{10000} & \multirow{2}{*}{1000} & normal & OOOOOOO \\
                       &                       & livre  & OOOOOOO \\
\hline
\end{tabular}
\caption{Testes - Validação 1}
\label{tab:tv1}
\end{table}

\subsection{Validação 2 - Problema com solução conhecida}
Neste experimento, deve-se determinar a solução aproximada para
$u(x,y)$ em $\Omega = (0,1) \times (0,1)$ considerando na Eq. 
\eqref{eq:transporte}:
\begin{align}\label{eq:v2}
k &= 1\nonumber \\
\beta_x(x,y) &= 1\nonumber \\
\beta_y(x,y) &= 20y\nonumber \\
\gamma(x,y) &= 1\nonumber \\
f(x,y) \text{ tal que } u(x,y) &= 10xy(1-x)(1-y)e^{x^{4.5}}
\text{ é a solução exata }
\end{align}

e sabendo que $u(x,y) = 0$ no contorno de $\Omega$.

Veja na tabela \ref{tab:tv2} os tempos de execução deste experimento
por valores $n$ e $m$ e versão do algoritmo SOR.

\begin{table}[ht]
\centering
\begin{tabular}{|c|c|l|c|}
\hline 
\textbf{n} & \textbf{m} & \textbf{SOR} & \textbf{Tempo} \\
\hline
\multirow{2}{*}{25}    & \multirow{2}{*}{100}  & normal & OOOOOOO \\
                       &                       & livre  & OOOOOOO \\
\hline
\multirow{2}{*}{100}   & \multirow{2}{*}{100}  & normal & OOOOOOO \\
                       &                       & livre  & OOOOOOO \\
\hline
\multirow{2}{*}{500}   & \multirow{2}{*}{1000} & normal & OOOOOOO \\
                       &                       & livre  & OOOOOOO \\
\hline
\multirow{2}{*}{10000} & \multirow{2}{*}{1000} & normal & OOOOOOO \\
                       &                       & livre  & OOOOOOO \\
\hline
\end{tabular}
\caption{Testes - Validação 2}
\label{tab:tv2}
\end{table}

\subsection{Aplicação Física 1 - Resfriador bidimensional}
Este experimento consiste em uma aplicação dos métodos em questão para
resfriar uma massa aquecida. Exemplos podem incluir o resfriamento de
chips de computadores ou amplificadores elétricos. O modelo matemático
que descreve a tranferência de calor nas direções $x$ e $y$ é dado por:
\begin{equation} \label{eq:a1}
- k\left(\frac{\partial^2 u}{\partial x^2} + \frac{\partial^2 u}{\partial 
y^2}\right) + \frac{2c}{T}u = \frac{2c}{T}u_{ref} = 0 \text{   em   }
\Omega = (0,L) \times (0, W)
\end{equation}

\noindent onde $k$ é a condutividade térmica constante, $c$ é o coeficiente de
transferência de calor, $T$ é a altura do resfriador e $u_{ref}$ é a
temperatura de referência. Deve-se encontrar a temperatura no interior
do resfriador considerando as seguintes condições de contorno:
\begin{align*}
u(x,0) &= 70\\
u(x,W) &= 70\\
u(0,y) &= 200\\
k\frac{\partialu}{\partialn}(L,y) &= c(u_{ref} - u(L,y))
\end{align*}

Veja na tabela \ref{tab:taf1} os tempos de execução deste experimento
por valores $n$ e $m$ e versão do algoritmo SOR.

\begin{table}[ht]
\centering
\begin{tabular}{|c|c|l|c|}
\hline 
\textbf{n} & \textbf{m} & \textbf{SOR} & \textbf{Tempo} \\
\hline
\multirow{2}{*}{25}    & \multirow{2}{*}{100}  & normal & OOOOOOO \\
                       &                       & livre  & OOOOOOO \\
\hline
\multirow{2}{*}{100}   & \multirow{2}{*}{100}  & normal & OOOOOOO \\
                       &                       & livre  & OOOOOOO \\
\hline
\multirow{2}{*}{500}   & \multirow{2}{*}{1000} & normal & OOOOOOO \\
                       &                       & livre  & OOOOOOO \\
\hline
\multirow{2}{*}{10000} & \multirow{2}{*}{1000} & normal & OOOOOOO \\
                       &                       & livre  & OOOOOOO \\
\hline
\end{tabular}
\caption{Testes - Aplicação Física 1}
\label{tab:taf1}
\end{table}

\subsection{Aplicação Física 2 - Escoamento em Águas Subterrâneas}
\lipsum[9]

Veja na tabela \ref{tab:taf2} os tempos de execução deste experimento
por valores $n$ e $m$ e versão do algoritmo SOR.

\begin{table}[ht]
\centering
\begin{tabular}{|c|c|l|c|}
\hline 
\textbf{n} & \textbf{m} & \textbf{SOR} & \textbf{Tempo} \\
\hline
\multirow{2}{*}{25}    & \multirow{2}{*}{100}  & normal & OOOOOOO \\
                       &                       & livre  & OOOOOOO \\
\hline
\multirow{2}{*}{100}   & \multirow{2}{*}{100}  & normal & OOOOOOO \\
                       &                       & livre  & OOOOOOO \\
\hline
\multirow{2}{*}{500}   & \multirow{2}{*}{1000} & normal & OOOOOOO \\
                       &                       & livre  & OOOOOOO \\
\hline
\multirow{2}{*}{10000} & \multirow{2}{*}{1000} & normal & OOOOOOO \\
                       &                       & livre  & OOOOOOO \\
\hline
\end{tabular}
\caption{Testes - Aplicação Física 2}
\label{tab:taf2}
\end{table}

% ----------------------------------------------------------
% Conclusão
% ----------------------------------------------------------
\section{Conclusão}
\lipsum[5]

% ----------------------------------------------------------
% Referências bibliográficas
% ----------------------------------------------------------
%\bibliography{trab2}{}
%\bibliographystyle{ieeetr}
%\addcontentsline{toc}{section}{Referências}

\end{document}
